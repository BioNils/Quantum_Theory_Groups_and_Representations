%!TEX TS-program = lualatex
\documentclass[11pt,a4paper]{article}
\usepackage[english]{babel} 
%LTeX: language=en-US
\usepackage[no-math]{fontspec}
%\usepackage{stmaryrd}
\usepackage{cite}
%\SetSymbolFont{stmry}{bold}{U}{stmry}{m}{n}
% margin
\setlength{\topmargin}{0.4cm}
\setlength{\headheight}{0.4cm}
\setlength{\headsep}{0.8cm}
\setlength{\footskip}{1cm}
\setlength{\textwidth}{17cm}
\setlength{\textheight}{24cm}
\setlength{\voffset}{-1.5cm}
\setlength{\hoffset}{-0.5cm}
\setlength{\oddsidemargin}{0cm}
\setlength{\evensidemargin}{0cm}
\usepackage{lipsum} 
\usepackage[]{mdframed}
\usepackage{physics} 
\usepackage{siunitx} % units
\DeclareSIUnit{\octet}{o}
\usepackage{svg}
\svgsetup{
    inkscape=pdf
}
\RequirePackage{lastpage}
\usepackage{appendix}
\usepackage{cite}
\usepackage{float}
\usepackage{multicol}
\usepackage{graphicx}			
\usepackage{amsmath}				
\usepackage{amssymb}			
\usepackage{amsfonts}
\usepackage{slashed}
\usepackage{subcaption}
\usepackage{fourier-orns}          		
\usepackage[T1]{fontenc}
\usepackage{tabularx}				
\usepackage{stmaryrd} 
\usepackage{psfrag}	
\usepackage[strict]{changepage}
\usepackage{amsthm}
\newtheorem*{theorem}{Theorem}
\newtheorem*{remark}{Remark}
\theoremstyle{definition}
\newtheorem{definition}{Definition}[section]
\makeatletter
\DeclareSizeFunction{sub}{\sub@sfcnt\@font@info}
\makeatother
\usepackage{color}
\usepackage{cancel}			
\usepackage{tabularx}
\usepackage{booktabs}
\usepackage[labelfont=bf,format=plain,justification=raggedright]{caption}
\definecolor{linkcolor}{rgb}{0,0,0.6}		
\usepackage[colorlinks=true,	
			pdfstartview=FitV,
			linkcolor= linkcolor,
			citecolor= linkcolor,
			urlcolor= linkcolor,
			hyperindex=true,
			hyperfigures=false,
			linktocpage=true,
			pdfproducer=medialab,
			pdfa=true
			]
			{hyperref}				
\usepackage{fancyhdr}
\usepackage{tikz}
\usetikzlibrary{arrows.meta,bending,positioning}
\usepackage[compat=1.1.0]{tikz-feynman}	
\usepackage[most]{tcolorbox}
\definecolor{ShadowColor}{RGB}{30,150,190}
\makeatletter
\newcommand\Cshadowbox{\VerbBox\@Cshadowbox}
\def\@Cshadowbox#1{%
  \setbox\@fancybox\hbox{\fbox{#1}}%
  \leavevmode\vbox{%
    \offinterlineskip
    \dimen@=\shadowsize
    \advance\dimen@ .5\fboxrule
    \hbox{\copy\@fancybox\kern.5\fboxrule\lower\shadowsize\hbox{%
      \color{ShadowColor}\vrule \@height\ht\@fancybox \@depth\dp\@fancybox \@width\dimen@}}%
    \vskip\dimexpr-\dimen@+0.5\fboxrule\relax
    \moveright\shadowsize\vbox{%
      \color{ShadowColor}\hrule \@width\wd\@fancybox \@height\dimen@}}}
\makeatother
\makeatletter
\@ifpackageloaded{babel}%
        {\newcommand{\nospace}[1]{{\NoAutoSpaceBeforeFDP{}#1}}}
        {\newcommand{\nospace}[1]{#1}}
\makeatother
\newcommand{\drawat}[3]{\makebox[0pt][l]{\raisebox{#2}{\hspace*{#1}#3}}}

\AtBeginDocument{\RenewCommandCopy{\qty}{\SI}}
\setlength{\headheight}{28.3 pt}
\numberwithin{equation}{section}
\newcommand{\pd}[1]{\partial_{#1}}
\newcommand{\pu}[1]{\partial^{#1}}
%%%%%%%%%%%%%% Color Macro
\usepackage{xcolor}
\makeatletter
\def\mathcolor#1#{\@mathcolor{#1}}
\def\@mathcolor#1#2#3{%
  \protect\leavevmode
  \begingroup
    \color#1{#2}#3%
  \endgroup
}
\makeatother
\title{Quantum Theory, Groups and Representations}
\begin{document}
\maketitle
\begin{definition}
    A representation $\left(\pi,V\right)$ of a group $G$ over a vector space $V$ is an application $\pi:G\rightarrow GL(V)$ such that $\forall g_1,g_2 \in G, \ \pi\left(g_1\cdot g_2\right) = \pi\left(g_1\right) \pi(g_2)$.
    \begin{itemize}
        \item A representation $(\pi,V)$ is said \textit{irreducible} if there is no proper subset $W \subset  V$ such that $\pi_W:G\rightarrow GL(W)$ is a representation.
        \item Otherwise, $(\pi,V)$ is said \textit{reducible}.
    \end{itemize}
\end{definition}
\begin{definition}
    A unitary representation $(\pi,V)$, where $V$ is a $\mathbb{K}$-EV equiped with an inner product $\expval{\ ,\ }:V\cross V \rightarrow \mathbb{K}$, is a representation which preserves the inner product such that:
    \begin{equation*}
        \forall g \in G,\ \forall v,w \in V,\ \expval{v,w} = \expval{\pi(g)v, \pi(g)w}. 
    \end{equation*}
    If $V$ is finite dimensionnal (of dimension $n$), we can treat the representation as an element of $\mathcal{M}_{n \cross n}\left(\mathbb{K}\right)$, and for a unitary transformation, $\forall g\in G, \pi(g) \in U(n)$.
\end{definition}
\begin{definition}
    (Direct sum representation). Given representations $\pi_1$ and $\pi_2$ of dimensions $n_1$ and $n_2$, there is a representation of dimension $n_1+n_2$ called the direct sum of the two representations, denoted $\pi_1 \oplus \pi_2$. This representation is given by the homomorphism:
    \begin{equation*}
        \left(\pi_1 \oplus \pi_2\right):g \in G \mapsto \begin{pmatrix}
            \pi_1(g)& \vb{0}\\
            \vb{0}& \pi_2(g)
        \end{pmatrix}.
    \end{equation*}
\end{definition}
\begin{theorem}
    A unitary transformation $\pi$ over a finite dimensionnal vector space $V$ can be written as $\oplus \pi_i $, where $\left\{\pi_i\right\}$ is a set of irreducible representations.
\end{theorem}
\begin{proof}
    If $(\pi,V)$ is irreducible, then it's obvious.\\
    Otherwise, there exists $W \subset V$ such that $(\pi_W,W)$ is a representation.  \\
    Since $V = W  \oplus W^\perp$, we want to check if $(\pi_{W^\perp},W^\perp)$ is a representation.
    \begin{itemize}
        \item Let $w \in W$ and $v \in W^\perp$, then obviously $v \perp w \Leftrightarrow \expval{v,w} = \expval{\pi(g)v,\pi(g)w}= 0$. Since $(\pi_W,W)$ is a representation, $\pi(g)w \in W$ and thus $\forall g \in G,\ \pi(g)v \perp W $. \\
        This means that $\forall g \in G,\ \forall v \in W^\perp,\ \pi(g)v \in W^\perp$. Thus, $\pi_{W^\perp},W^\perp$ is a representation.(It respects the group product since $(\pi,V)$ is a representation.)
    \end{itemize}
    Thus:\begin{equation*}
        (\pi,V) = (\pi_W,W)\oplus(\pi_{W^\perp},W^\perp).
    \end{equation*}
    We can then repeat this process until each subrepresentation is irreducible.
\end{proof}
\begin{theorem}{Schur's Lemna:}
    If a \textbf{complex} representation $(\pi,V)$ is irreducible, then the only application $M:V\rightarrow V$ commuting with all the $\pi(g)$ is $M = \lambda \mathbb{I},\ \lambda \in \mathbb{C}$. 
\end{theorem}
\begin{proof}
    Let's assume that $(\pi,V)$ is an irreducible representation and that $\forall g \in G, \left[M,\pi(g)\right] = 0$. Since $V$ is a $\mathbb{C}$-vector space, the equation:
    \begin{equation*}
        \det(M-\lambda \mathbb{I}) = 0
    \end{equation*}
    has $n$ roots (counting the multiplicity) and thus we can find all the eigenspaces:
    \begin{equation*}
        U_\lambda = \left\{v\in V | Mv = \lambda v\right\} = \ker\left(M-\lambda \mathbb{I}\right) \neq  \emptyset .
    \end{equation*}
    Since $\pi$ and $M$ commute, $v\in\ker\left(M-\lambda \mathbb{I}\right) \implies  \forall g \in G\ \pi(g)v \in \ker\left(M-\lambda \mathbb{I}\right) $.
    As a consequence, $(\pi,\ker\left(M-\lambda \mathbb{I}\right))$ is a representation of $G$ since $\ker\left(M-\lambda \mathbb{I}\right)$ is stable under the action of $\pi$.
    Since we supposed that $(\pi,V)$ is irreducible then either:
    \begin{itemize}
        \item $U_\lambda = \ker\left(M-\lambda \mathbb{I}\right) = \emptyset$, but it's not possible since $U_\lambda$ is an eignespace.
        \item $U_\lambda = \ker\left(M-\lambda \mathbb{I}\right) = V \Leftrightarrow \forall v \in V, (M-\lambda \mathbb{I})v=0$, which implies that $M = \lambda \mathbb{I}$.
    \end{itemize}
\end{proof}
\begin{theorem}
    If $G$ is an abelian group, then all of its irreducible representations are one dimensionnal.
\end{theorem}
\begin{proof}
    Consider an abelian (=commutative) group $G$ and $g \in G$.
    Any representation $(\pi,V)$ of $G$ will satisfy:
    \begin{equation*}
       \forall h \in G,\  \pi(g) \pi(h) = \pi(h) \pi(g).
    \end{equation*}
and if $\pi$ is irreducible, by using Schur's Lemna, $\pi(h) = \lambda \mathbb{I} \simeq \lambda$ for $\lambda \in \mathbb{C}$.
\end{proof}
\end{document}
